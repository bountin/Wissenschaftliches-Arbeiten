\documentclass{"../../Datenanalyse/assignments/latex-brookesassignment/brookes-assignment"} 

\usepackage[english]{babel}
\usepackage[utf8]{inputenc}

	% set some of the assignment details
	\author{Martin Prebio}
	\title{Wissenschaftliches Arbeiten}
	\studentnumber{1025737}
	\subtitle{Review Log}
	\modulecode{184.690}
	\modulename{Wissenschaftliches Arbeiten}
	\assignmentname{Vienna University of Technology}

\begin{document}
\maketitle

% TODO: Fill with introduction
% Number in parentheses -> Review Document (1 Unnamed english, 2 german, 3 by Mr Scharinger)

\section{Language \& Style}
\subsection{However sometimes either the wording is confusing or I do not get the message right. E.g. ``Another fairness criteria could be that the minimum utility is maximized.'' (1)}
Rephrased and extended.
\subsection{Another point is that it is more common to use cannot instead of `can not' and another instead of `an other'. (1)}
Searched and Replaced.
\subsection{Im Großen und Ganzen in Ordnung. Ein Zeitfehler in Kapitel 3, letzter Absatz: ``From this set they chose that'', hier gehört choose(Gegenwart) (2)}
Corrected typo.
\subsection{In section 4 Bidding Representations \& Languages the article says ``To be equivalent to the naive form these atomic bid must also be OR-conjunctional so that a bidder can submit several bundle he or she wants to bid for.'' The word bundle is missing the -s ending for plural. (3)}
Corrected typo.
\subsection{Use of correct/better brackets in section 3 instead of less-than and greater-than signs. (3)}
Replaced all tupel brackets with $\langle$ and $\rangle$. Also fixed two missing set brackets.
\subsection{On page six the paper says ``During the previous section some small examples were already given but the following ones will go a bit deeper into real work applications.'' I think there is a typo and it should say real world applications. (3)}
Corrected auto correction.

\section{Topic Mistakes}
\subsection{On page five the article says ``[BCE12, pp. 46–47] shows that this `Winner Determination Problem' can be converted into the `Set Packing Problem' which is known to be NP-hard.'' But [BCE12] reduces ``Set Packing Problem'' to ``Winner Determination Problem'' in order to show NP-hardness of the ``Winner Determination Problem''. That is, the other way around. Furthermore he states that it is NP-complete. (3)}
Corrected.

\section{Clarity}
\subsection{In Kapitel 1.2, letzter Satz geht nicht ganz klar hervor was ``ordinal preference'' bedeutet. (2)}
Extended the example a bit.
\subsection{In the restaurant example in section 1.5 Combinatorial Domains it is not clear at first glance that the author does not use a combinatorial voting system. (3)}
Redefined the example a bit by stripping not needed parts and reworked the text in general.
\subsection{In section 4 Bidding Representations \& Languages I think the part about XOR-bids can benefit from some more explanations. I feel that it would be easier to understand, if the the dummy items would have been explained further. [FLBS99] actually has a better example.}
Added another sentence for explanation.
\subsection{On page five the article says ``Its upper limit is $O(n^n)$.'' without saying what n is. (3)}
Added.

\section{References \& Literature}
\subsection{Sometimes only the page numbers are given (without the word `pages'), sometimes it is `pages 1-12' (1)}
The inconsistency comes from Google Scholar where different BibTex classes (in proceedings, book, article) are used which appears to render differently. I did not change this to stick to Scholar's output, and live with this inconsistency.
\subsection{In Kapitel 1.1 wäre eine Referenz wünschenswert, ansonsten ausreichend Literatur referenziert. (2)}
From my point of view this is widely known and so a reference is not necessary. Alternatively more information can be found in the references used in the following sub sections.
\subsection{On page six the paper says ``[CELM07] had a look at the American truckload transportation market during the late 1990s and early 2000s.'' I believe the reference is wrong and should be replaced by [CS06]. The bibliography entry ``[CS06]'' is not referenced anywhere in the article. (3)}
Corrected.

\section{Further Suggestions}
\subsection{Table of Contents (2)}
\subsection{Greater outlook (1)}
\subsection{More in-depth (1)}
\subsection{I would have liked it even more if the conclusion had been a bit longer to summarize the whole article once more (and not just in short terms) since I prefer to have a good overview over the things I have already read so that it sticks in my mind better. (1)}
I have extended the closing words.

\end{document}
